\documentclass{article}

\usepackage[utf8]{inputenc}
\usepackage{graphicx} % Required for inserting images
\usepackage{amsmath}
\usepackage{amsfonts}
\usepackage{amssymb}

% Manually set page margins if needed
\usepackage{geometry}
\geometry{a4paper, margin=1in}

\begin{document}

\noindent\makebox[\linewidth]{\rule{\textwidth}{0.4pt}}

\begin{center}
{\Large Università di Padova - Dipartimento di Fisica e Astronomia} \\[1em]
{\large Corso: Sperimentazioni di Fisica 1 - Canale M-Z.} \\
{\large Anno accademico: 2022-23. Docenti: D. Mengoni, M. Doro} \\[2em]

Gruppo XX \\
Emilio Manzotti - Nr. Matricola - emilio.manzotti@studenti.unipd.it \\
Greta Spanò - Nr. Matricola - greta.spano@studenti.unipd.it \\
Marilena Santarossa - Nr. Matricola - marilena.santarossa@studenti.unipd.it \\[1em]
Data: 10/12/2023
\end{center}

\noindent\makebox[\linewidth]{\rule{\textwidth}{0.4pt}}

\vspace{2em}

\section*{Stima della accelerazione di gravità con misure ripetute di periodo di oscillazione di un pendolo semplice}
% write an introduction


\section{Indagine del problema fisico e Ideazione di una misura: Obiettivo e metodologia dell’esperienza}
% [Max 2-3 pagine] 
% Descrivere in breve gli obbiettivi della esperienza. Descrivere in modo sintetico il misurando e il metodo della misura. Individuare a priori le possibili fonti di incertezza ed errore associate alla misura.


\section{Realizzazione e messa in opera dell’apparato. Campagne di presa e analisi dati di controllo.}
% [Max 5-6 pagine] 
% Descrivere brevemente l’apparato realizzato, le soluzioni utilizzate. Descrive in dettaglio i test svolti per la valutazioni della precisione dell’apparato e della valutazione degli errori sistematici nello strumento, nel modello di riferimento, nella misura e nella lettura. In particolare rispondere a queste domande: Il modello matematico del pendolo semplice e’ accurato entro la precisione sperimentale? Il pendolo smorza l’oscillazione in maniera significativa? Di quanto? Quale pu`o essere la causa? Influisce sulla stima del misurando? Il pendolo rallenta in maniera significativa? Di quanto? Quale puo’ essere la causa? Influisce sulla stima del misurando?

\subsection{Test}

\section{Rapporto sui risultati principali. Presentazione grafica e numerica risultati}
% [Max 2 pagine] 
% In questa parte si va al sodo a riportare la stima migliore del misurando, si puo’ fare riferimento, nella discussione sulla precisione e accuratezza, a considerazioni fatte nella sezione precedente. In particolare rispondere a queste domande: Quali sono i fattori che influiscono maggiormente sulla precisione della misura? Quali sono i fattori che influiscono maggiormente sulla accuratezza della misura?

\section{ Conclusioni e Prospettive}
% [Max 1 pagina] 
% Poi si procede con la la discussione e il ragionamento sulle prospettiva: cosa avrei potuto fare diversamente per avere un risultato migliore? Che ulteriori test si possono fare? Come si puo’ migliorare lo strumento


\section{Appendici}
% [Massimo circa 5 pagine]
% In appendice si riporta tutto quello che in prima lettura non e’ necessario a dimostrare le conclusioni trovate durante l’esperienza, ma che pu`o tornare utile in caso di un controllo successivo o un esame approfondito. Ad esempio in appendice si possono riportare i codici usati, tabelle troppo lunghe, grafici di controllo. Anche in questo caso usare moderazione.

\end{document}
